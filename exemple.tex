\documentclass[aspectratio=169,professionalfonts, 12pt]{beamer}
\usepackage{lmodern}
\usepackage{array}
\usepackage{multirow}
\usepackage[french,english]{babel}
\usepackage[T1]{fontenc}
\usepackage{multicol}
\usepackage{ragged2e}   %new code
%%%
\usepackage[utf8]{inputenc}
\usepackage[brazil]{varioref}
\usepackage[square,sort,comma,super,authoryear]{natbib}
\usepackage{listings,xcolor}
\usepackage{xmpmulti}
\usepackage{epsfig}
\usepackage{subcaption}
\captionsetup{compatibility=false}
\usepackage{ru,graphicx,hyperref,url} % 
\usepackage{booktabs}
\usepackage{pgfplots}
\usepackage{tikz}
\usepackage{ amsmath, amssymb, amsfonts}
\setbeamertemplate{navigation symbols}{}
\addtobeamertemplate{block begin}{}{\justifying}
\setbeamertemplate{section in toc}[sections numbered]

\AtBeginSection[]
{
  \begin{frame}[t]
  \begin{multicols}{2}
      \tableofcontents[currentsection]
    \end{multicols}
  \end{frame}
}

\useoutertheme{infolines}
\setbeamertemplate{footline}{\hspace*{.5cm}\scriptsize{\insertshorttitle
\hspace*{50pt} \hfill\insertframenumber\hspace*{.5cm}}\\
\vspace{9pt}} 
\date{\today}

\definecolor{blueforest}{RGB}{80,00,00}

\begin{document}
  \selectlanguage{french}

\begin{frame}[plain]
  \titlepage
\end{frame}

  

  \begin{frame}[plain]{Table des matières}
    \tableofcontents
  \end{frame}

\section{Introduction}

\begin{frame}{Introduction}
  La logique floue a été créée en 1965 par Lotfi Zadeh, elle se base sur la théorie
  des ensembles flous et la logique. La logique floue est une méthode qui offre des grandes
  performances permettant de gérer des systèmes complexe de façon intuitive. Elle est
  une extension de la logique booléenne qui permet la modélisation des imperfections des données
  et se rapproche dans une certaine mesure de la flexibilité du raisonnement humain.
\end{frame}

\section{Différences entre la logique floue et logique classique}
\section{Le fonctionnement d’un système flou}
\section{Les domaines d’application de la logique floue}
\section{Les avantages et inconvénient de la logique floue}
\section{Conclusion}

\end{document}
